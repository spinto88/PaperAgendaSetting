\section{Context}
\label{sec:Context}

\subsection{\emph{Elections}}
\par Two legislative elections were celebrated during the period in great part of the Argentina: Primary elections on August 13th and the general elections on October 22th. 
A special focus was put on the elections in the Buenos Aires province, where the former President \emph{Cristina Fernández de Kirchner} participated as a senator candidate representing the alliance \emph{Unidad Ciudadana}, confronting to the alliance \emph{Cambiemos}, which is the alliance of the current President \emph{Mauricio Macri} and the current governor of Buenos Aires province, \emph{Maria Eugenia Vidal}.
 
\subsection{Current President}
\par \emph{Mauricio Macri} is the current Argentinian President, since December 2015. 
Most articles in political sections are logically devoted to him under different contexts.
However, it is important to point out that during the period analyzed, and specially after the general elections of October 22th, a controversial labour reform promoted by the government was been discussing.

\subsection{\emph{Missing Person}}
\par \emph{Santiago Maldonado} vanished on August 1st after a minor clash between the Gendarmerie (Border Guards) and a group of Mapuches, which recognize as themselves the original population of an area in the Patagonia.
Since that event, the \emph{Mauricio Macri}'s administration was accused by several people as the responsible of a \textbf{forced disappearance}. 
\par A very massive campaign in social media took place on August 26th, under the motto "Where is Santiago Maldonado?", followed by two massive protest marches to the \emph{Plaza de Mayo} took place on September 1st and October 1st, which the first one had a great repercussion due to several incidents that took place during the march.
\par The body of \emph{Santiago Maldonado} was found on 17th October in the \emph{Chubut} river, near the place where he was seen the last time, and the autopsy report told that \emph{Santiago Maldonado} had died from ‘asphyxia after being submerged,’ with no injuries on his body. 
However the responsability of the current administration is still being discussed.

\subsection{\emph{Former Planning minister} and \emph{Former Vice-President}}

\par \emph{Julio de Vido} was the Planning minister during the administration of \emph{Nestor Kirchner} and \emph{Cristina Fernandez de Kirchner} (2003-2015). In 2015, he was elected to integrate the Chamber of Deputies, which finally voted to strip \emph{De Vido} of his congressional immunity over corruption allegations and was inmmediatly jailed on October 27th.

\par \emph{Amado Boudou} was the Vice-President of the \emph{Cristina Kirchner}'s administration.
\emph{Boudou} was arrested on November 3th on charges including money-laundering and hiding undeclared assests.

\subsection{\emph{Social leader} and \emph{Prosecutor's death}}

\par \emph{Milagro Sala} is an indigenous leader’s. 
She has been incarcerated under pre-trial detention ever since she was first detained in January 2016. She faces allegations of embezzlement related to government funding for housing projects managed by Túpac Amaru, her social organization.
Sala accused the government of "violating her human rights", and several people think that she is a political prisoner of the \emph{Mauricio Macri} administration.

\par \emph{Alberto Nisman} was a special prosecutor who were investigating the 1994 terror attack on the Argentine Israeli Mutual Association (AMIA), until his suspicious death on January 2015.
During the period analyzed in this work, a team of experts led by the Gendarmerie (Border Guard) concluded that late prosecutor's death may have been a case of murder, not suicide.
