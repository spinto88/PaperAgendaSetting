
%PRIMERA PARTE TEORIA DE AGENDA SETTING
\subsection{Agenda Setting Theory}

% Introducción y descripción de agenda-setting en su nivel más básico.
\par In the famous study performed in Chapel Hill during the US presidential elections in 1968 \cite{mccombs1972agenda}, Maxwell McCombs and Donald Shaw found that those aspects of public affairs that are prominent in the news become prominent among the public.
This study is considered the founding of the agenda-setting theory, which focus in the influence of mass media in public opinion.  
From \cite{mccombs2014agenda}, \textit{``The media agenda is the pattern of news coverage over a period of days, weeks (...) for a set of issues or other topic. In other words, the media agenda is a systematic compilation of the issues or topics presented to the public that identifies the degree of emphasis on these topics.''}

\par Since the Chapel Hill research, several directions of agenda-setting were established \cite{mccombs2005look}.
In its basic stage, the theory is focus on the comparison between the topics coverage by the media and the public agenda, i.e. the topics that the public consider as priority.
It looks for answering the question if the media is able to set the public agenda, which would transform the media as an important actor in the formation of public opinion. The interaction between media and public is rather complex, for instance, \cite{mitchelstein2016brecha} shows that not necessarily the journalists and public preferences coincide.
\par With the emergency of the Internet, the end of agenda-setting were predicted due to the audience fragmentation onto multiple sources, which would virtually lead to a highly individualized agenda.
However, it is based on two assumptions that not necessarily are true: that the public spreads its attention in an homogeneous way across the multiple sources, and that the agendas of that sources are different \cite{mccombs2005look}.

% Agenda setting de segundo nivel: noticias con atributos y contextualización.
\par The basic agenda-setting sometimes is called the ``first level agenda-setting''. 
The very often quoted phrase of Bernard C. Cohen \textit{``The press may not be successful much of the time in telling people what to think, but it is stunningly successful in telling its readers what to think about.''} illustrates its object of study.
On the other hand, the ``second level agenda-setting'', sometimes called \textit{attribute agenda-setting}, studies the \textit{objects} (in a social psychology way, where an \textit{attitude object} designate a thing that an individual has an attitude or opinion about) present in the media agenda. When the media talks about an object some attributes are emphasized, and others not. 

\par The ``second level agenda-setting'' is linked with \textit{framing} \cite{guggenheim2015dynamics} \cite{tsur2015frame}. 
To frame is to \textit{select some aspects of a perceived reality and make them more salient in a communicating text, in such a way as to promote a particular problem definition, causal interpretation, moral evaluation and/or treatment recommendation} (Robert Entman) \cite{mccombs2005look}.
This stage of agenda-setting theory can be summary in the phrase \textit{``the media not only can be successful in telling us what to think about, they also can be successful in telling us how to think about it.''} 

% Intermedia setting-agenda: compentencia entre medios de comunicación.
\par Other interesting stage of agenda-setting concerns with the sources of media agenda, i.e. if the media set the public agenda, \textit{who sets the agenda media?} 
Within this framework, \textit{intermedia agenda-setting} observes the competition between different media and how they influence each other.
The competition between mass media for the same audience can lead to a homogenization of the agendas \cite{vargo2017networks}, which is in the opposite direction of one of the assumptions that predict the end of agenda-setting, as was mentioned before.


% ACA ARRANCA LA REESTRUCTURACION NUEVA
\subsection{Related works}

\par Typical works in agenda-setting theory in any of its stages mimics the original work of McCombs \cite{mccombs1972agenda}.
The study of the agenda is usually done in a static way, i.e., by summarizing certain issues present in news media and looking for effects in the audience \cite{brians1996campaign}\cite{gerber2009does}\cite{coleman2007young}.
For instance, in Argentina, \cite{mitchelstein2016brecha} examines the difference between public and journalists preferences. 
Several works by E. Zunino, like \cite{zunino2010cobertura} and \cite{koziner2013cobertura}, studied the coverage given by the main newspapers of particular events which have happened during the administration of Argentina’s president \emph{Cristina Fernández de Kirchner} (from 2007 to 2015), where the government started to confront with news organizations that were critical to its management as they were an opposition party \cite{mitchelstein2017information}

% Algo sobre bias
\par Other typical works are concerned with detecting bias in the media, by focusing in detecting its ideology, either by taking into account the number of mentions of a preferred political party \cite{lazaridou2016identifying}\cite{baumgartner2015all} or by identifying the ideology through the position of the media respect to certain issues or actors \cite{elejalde2018nature}\cite{sagarzazu2017hugo}.
New trends in agenda-setting theory propose to represent ideological aspects or issues emphasized as networks like mind maps, and the basic idea is the comparison between media's knowledge representative network with its counterpart in the audience \cite{guo2012application}\cite{vu2014exploring}.

\par Most work cited above follow usual methods employed in social science research.
However, a very useful tool in the analysis of large corpus of documents which is not widely used in the agenda setting theory is unsupervised topic modeling.
It is an alternative to the dictionary-based analysis, which is the most popular automated analysis approach \cite{guo2016big}, and allows to work with a corpus without a prior knowledge, letting the topics emerge from the data. Despite the popularity of this methods, we believe that there is still a lack in the employment of these ones through the lens of the agenda setting framework.
Many works based on news corpus emphasize the performance of the topic model over a labeled corpus, focusing on the proper detection of the topics \cite{dai2010online}\cite{po2016topic}\cite{brun2000experiment}.
The temporal profile of topics is usually embedded in the context of topic tracking \cite{hu2016news}\cite{li2017joint}, or in the recognition of emerging topics in real-time \cite{cataldi2010emerging} mostly applied to social media.
\par Typically a dynamical description is not carried out in a set of topics but rather focused on a single issue.
For instance, in \cite{soroka2017negativity} it is shown that the newspapers and Twitter have an opposite reaction to the changes of the unemployment rates, in \cite{guggenheim2015dynamics} the competition of frames in this case about gun control is explored,
and in \cite{ali2018measuring} it is shown how twitter activity varies in different regions depending on the location of terrorist attacks.
A remarkable exception is \cite{russell2014dynamics} where they work with a set of predefined issues. In this work the question of \emph{causality} is also faced up. Their study shows that sometimes the traditional media set the agenda and sometimes, the social media does. They show that social media is always more interested in social issues than the traditional one, and despite the existence of correlation, the social media agenda can not be seen as a \emph{slave} of the traditional media.

% Presentación y descripción de nuestro trabajo
\par In this work we propose a simple method to the study of Mass Media and audience response through topic detection algorithms.
Our work intends to contribute another quantitative approach which complements the agenda-setting theory describe above.
Rather than focus on a single issue or a set of independent topics, we work with the agendas (the media and the public) as an object in their own, studying their evolution over time.
On the other hand, we aim to take an insight about Media dynamics and Public response in order to create useful tools at the time of constructing mathematical models about the interaction between Media and Public, investigation that we started in \cite{pinto2016setting}.


